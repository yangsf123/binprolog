\documentclass{article}

\textheight 21 cm
\textwidth 14 cm
 
\topmargin -1mm
\marginparwidth 0mm
\evensidemargin 1.5 cm
\oddsidemargin 1.5 cm

\begin{document}
\pagestyle{plain}

\bibliographystyle{abbrv}

%\newtheorem{df}{Definition}
%\newtheorem{th}{Theorem}

\title{
\Huge
BinProlog 2006 11.x Professional Edition\\
BinProlog Interface Guide\\
Connecting BinProlog with C/C++, Java, Tcl/Tk %, HTML, CGIs and VRML
\vskip 5cm
}


\author{
\Large
   {\bf Paul Tarau}\\\\
\large
   BinNet Corp.\\
   WWW: http://www.binnetcorp.com\\
   {E-mail: binnetcorp@binnetcorp.com}
\normalsize
}

%\date{}

\vskip 5cm
\maketitle

\newpage



\section{BinProlog's C-interface}

To be able to extend BinProlog and possibly embed it as logic
engine in your no run-time fee application you will need
a BinProlog C-source code
license. See the files SOURCE.LICENSE and PRICING for more information.
You can also interact with other languages under UNIX using the
bidirectional pipe based interface starting from BinProlog's Tcl/Tk
interfaced (see directory TCL).

The following sequence of quick examples shows how it works.

\subsection{A Demo Program}

BinProlog's C-interface together with a prototype example been moved in
directory {\tt c\_inter}. Start by cloning this directory if you
want to build an interface application and edit/adapt the files.

The discussion which follows basically explains
the content of the files {\tt c.c c.h and c.pl} in this directory.

Once BinProlog is correctly installed as file "bp" somewhere in
your path along with the libraries bp.o and wam.o which should
be in directory
{\tt ../src} and you have a C-compiler around, you can simply
do {\tt make} in directory {\tt c\_inter} to create a standalone application 
which integrates the functionality of the C-code in file c.c.

To test the resulting standalone executable {\tt c\_binpro}, 
try out the demo {\tt c\_test}
defined in file {\tt c.pl}.
Start running it by typing {\tt c\_binpro}
and then type {\tt c\_test.} at the
Prolog prompt. To avoid the prompt and directly launch
the application add 
\begin{verbatim}
main:-c_test.
\end{verbatim}
in file {\tt c.pl} before typing {\tt make}.


\subsection{Calling C from BinProlog: adding new builtins}

New builtins are added in {\tt headers.pl} and after a "make realclean; make"
a new version of BinProlog containing them is generated automatically.
Starting from {\tt headers.pl} the system will first generate the files
{\tt defs.h, prof.h, builtins.pl} and then recompile itself.
For this recompilation either a previous version of BinProlog\footnote{Use
"make remake or make again" if BinProlog is the only Prolog around.}

or SICStus Prolog 2.x is used.

An example of declaration in {\tt headers.pl} is:

{\small
\begin{verbatim}
  b0(+/3,arith(1),in_body).  <=== arity=3+1 by binarization
\end{verbatim}
}

{\flushleft Notice} that arith(1) means that it is
 "like an arithmetic functions" which returns {\em one} value, i.e this should be used as in

{\small
\begin{verbatim}
  +(0,1,Result).
\end{verbatim}
}

{\flushleft I would} suggest to start with the simplest form of interaction:
a call of your own C-code from BinProlog. Try modifying in the file
c.c (provided with the C-sources of BinProlog,
in directory {\tt c\_inter}),
the function {\tt new\_builtin()} which looks as follows:

{\small
\begin{verbatim}
/* ADD YOUR NEW BUILTINS HERE

   they can be called from Prolog as:
   new_builtin(0,<INPUT_ARG>,<OUTPUT_ARG>)
   
   X(1) contains the integer `opcode' of your builtin
   X(2) contains your input arg
   regs[I] contains somthing that will be unified with what you return
      (the precise value of I depends on the register allocator).

   You are expected to `return' either 
    - a non-null object that will be unified with <OUTPUT_ARG>, or
    - NULL to signal FAILURE

   As the returned object will be in a register this
   can be used for instance to add a garbage collector
   that moves every data area around...

*/

/**
  simple function to be called from BinProlog
*/
int sfun(char *s) {
   char buf[100];
   sprintf(buf,"=> CALLED sfun(char *) at %ld!",(unsigned long)sfun);
   strcat(s,buf);
   return 1;
}

/**
  simple float function to be called from BinProlog
*/
double ffun(double d) {
   char buf[100];
   sprintf(buf,"=> CALLED double<-ffun(double d) at %ld!",(unsigned long)ffun);
   return d*2.5;
}

term new_builtin(register term H,register term regs,register term *A,register instr P,register stack wam) {
   BP_check_call(); 
   switch(BP_op)
  { 
    /* for beginners ... */

    case 0: 
       /* this just returns your input argument (default behavior) */
    break;

    case 1: 
      BP_result=BP_integer(13); /* this example returns 13 */
    break;

    case 2: 
      BP_result=BP_atom("hello"); /* this example returns 'hello' */
    break;

    /* for experts ... */

    case 3: /* iterative list construction */
    { cell middle,last,F1,F2; int i;
      BP_make_float(F1, 77.0/2);
      BP_make_float(F2, 3.14);

      BP_begin_put_list(middle);
         BP_put_list(BP_integer(33));
         BP_put_list(F1);
         BP_put_list(BP_string("hello"));
         BP_put_list(F2);
      BP_end_put_list();
 
      BP_begin_put_list(last);
        for(i=0; i<5; i++) {
          BP_put_list(BP_integer(i));
        }
      BP_end_put_list();
      
      BP_begin_put_list(BP_result);
        BP_put_list(BP_string("first"));
        BP_put_list(middle);
        BP_put_list(last);
        BP_put_list(F1);
        BP_put_list(F2);
      BP_end_put_list(); 
    } break;

    case 4: /* cons style list construction */
      BP_begin_cons();
      BP_result=
       BP_cons(
          BP_integer(1),
          BP_cons(
               BP_integer(2),
               BP_nil
          )
       );
      BP_end_cons();
    break;

    case 5: /* for hackers only ... */ ;
      BP_result=(cell)H;

      H[0]=g.DOT;
      H[1]=X(2);

      H[2]=g.DOT;
      H[3]=BP_integer(99);

      H[4]=g.DOT;
      H[5]=(cell)(H+5); /* new var */

      H[6]=g.DOT;
      H[7]=(cell)(H+5); /* same var as previously created */

      H[8]=g.DOT;
      H[9]=BP_atom("that's it");

      H[10]=g.NIL;
      H+=11;
    break;

    case 6:
      BP_fail();
    break;

    case 7: 
      {  cell T=BP_input;
         if(BP_is_integer(T))
            {int i;
             BP_get_integer(T,i);
             fprintf(g.tellfile,"integer: %ld\n",i);
             BP_result=BP_integer(-1);
            }
         else
            BP_fail();         
      } 
    break;

   case 8: /* for experts: calling BinProlog from C */
      { cell L,R,Goal;
        
        BP_begin_put_list(L);
          BP_put_list(BP_string("one"));
          BP_put_list(BP_integer(2));
          BP_put_list(BP_string("three"));
          BP_put_list(BP_input); /* whatever comes as input */
        BP_end_put_list();
  
        BP_put_functor(Goal,"append",3);
        BP_put_old_var(L);
        BP_put_old_var(L);
        BP_put_new_var(R);        

        BP_prolog_call(Goal); /* this will return NULL on failure !!!*/

        BP_put_functor(Goal,"write",1);
        BP_put_old_var(R);    

        BP_prolog_call(Goal); /* calls write/1 */

        BP_put_functor(Goal,"nl",0);

        BP_prolog_call(Goal); /* calls nl/0 */

        BP_result=R; /* returns the appended list to Prolog */

      }
      break;

    case 10: /* for experts: calling BinProlog from C */
      { cell L,R,Goal;
        
        BP_begin_put_list(L);
          BP_put_list(BP_string("one"));
          BP_put_list(BP_integer(2));
          BP_put_list(BP_string("three"));
          BP_put_list(BP_input); /* whatever comes as input */
        BP_end_put_list();
  
        BP_put_functor(Goal,"cut_test",3); /* see c.pl */
        BP_put_old_var(L);
        BP_put_old_var(L);
        BP_put_new_var(R);        

        BP_prolog_call(Goal); /* this will return NULL on failure !!!*/

        BP_put_functor(Goal,"write",1);
        BP_put_old_var(R);    

        BP_prolog_call(Goal); /* calls write/1 */

        BP_put_functor(Goal,"nl",0);

        BP_prolog_call(Goal); /* calls nl/0 */

        BP_result=R; /* returns the appended list to Prolog */

      }
      break;

   case 11: /* CALLING BinProlog on a NEW ENGINE */
      { stack Engine;
        cell Goal,Answer;
        cell Xs,X;
        
        BP_begin_put_list(Xs);
          BP_put_list(BP_string("new_engine_test"));
          BP_put_list(BP_input); /* whatever comes as input */
          BP_put_list(BP_integer(100));
          BP_put_list(BP_integer(200));
          BP_put_list(BP_integer(300));
        BP_end_put_list();
  
        BP_put_functor(Goal,"member",2);
        BP_put_new_var(X);        
        BP_put_old_var(Xs);

        Engine=BP_create_engine(wam,100,50,50);

        BP_load_engine(Engine,Goal,X);

        while((Answer=(cell)ask_engine(Engine)))
          { cell Goal1;

            BP_put_functor(Goal1,"write",1);
            BP_put_old_var(Answer);    

            BP_prolog_call(Goal1); /* calls write/1 */

            BP_put_functor(Goal1,"nl",0);

            BP_prolog_call(Goal1); /* calls nl/0 */

          }
        BP_destroy_engine(Engine);
        BP_result=Goal; /* returns the goal - just for tracing */
      }
      break;

  case 12: /* CALLING BinProlog on a NEW ENGINE. The friendliest way... */
      { stack Engine;
        cell Goal,Answer;
        cell Xs,X;
 
        BP_sread("[0,s(0),s(s(0)),[a,b,c]]",Xs);
         
        BP_put_functor(Goal,"member",2);
        BP_put_new_var(X);        
        BP_put_old_var(Xs);

        Engine=BP_create_engine(wam,100,50,50);

        BP_load_engine(Engine,Goal,X);

        printf("BP_input: = %s\n",BP_swrite(BP_input));  

        while((Answer=(cell)ask_engine(Engine)))
          { 
             printf("Answer = %s\n",BP_swrite(Answer));
          }
        BP_destroy_engine(Engine);
        BP_result=Goal; /* returns the goal - just for tracing */
      }
      break;


  case 13: /* CALLING BinProlog on a NEW ENGINE. The friendliest way... */
      { stack Engine; int i;
        cell Goal,Answer,Answers[4],Ys;
        cell Xs,X;
 
        BP_sread("[0,s(0),s(s(0)),[a,b,c]]",Xs);
         
        BP_put_functor(Goal,"member",2);
        BP_put_new_var(X);        
        BP_put_old_var(Xs);

        Engine=BP_create_engine(wam,100,50,50);

        BP_load_engine(Engine,Goal,X);

        /* accumulates the answers */

        for(i=0; Answer=(cell)ask_engine(Engine); i++)
            { 
              /* To survive failure in Engine !!!! 
                 After copying, Answer will point to a valid object on 
                 the calling engine's heap whose H is also updated (see
                 BP_copy_answer in c.h). 
              */
              BP_copy_answer(Answer);
              Answers[i]=Answer; 
            }
        BP_destroy_engine(Engine);

        /* prints out the accumulated answers */

        printf("BP_input: = %s\n",BP_swrite(BP_input));
        BP_begin_put_list(Ys);
        for(i=0; i<4; i++)
            { 
               printf("Answers[%d] = %s\n",i,BP_swrite(Answers[i]));
               BP_put_list(Answers[i]); /* to return them to Prolog */
            }
        BP_end_put_list();       
        BP_result=Ys; /* returns the list of answers, Ys,
                         after constructing it on the heap */
      }
      break;

      /**
        Calls a string transformer function from BinProlog:
        sfun gets a char * argument which it modifies at will.
        Signals succes with 1, failure with 0
      */
      case 14:
        BP_result=BP_funptr(sfun);
      break;
      
      /**
        Calls a C function with an int or float argument
        and returns a float result
      */
      case 15: {  
         double d;
         if(BP_is_integer(BP_input)) {
            int i;
            BP_get_integer(BP_input,i);
            d=(double)i;
         }
         else if(BP_is_atom(BP_input)) { // it means ref to a BP_FLOAT
            char *s=NULL;
            BP_get_string(BP_input,s);
            d=atof(s);
         }
         else if(BP_is_var(BP_input)) { // it means ref to a BP_FLOAT
            int ok;
            BP_get_float(BP_input,d,ok);
            if(!ok) {
              printf("not a float: = %s\n",BP_swrite(BP_input));  
              BP_fail();
            }
         }
         else  /* this returns with failure */ {
            printf("BAD BP_input: = %s\n",BP_swrite(BP_input));
            BP_fail();
         }
            
         /* compute results */   
         {  double r=ffun(d);
            BP_make_float(BP_result,r);
         }
      } 
      break;

      /*
        getting any simple (int,double,string) data from
        prolog - the C side trusts that Prolog sends
        the right data - which isconverted
        automatically based on its dynamic Prolog type
      */
      case 16: {
       if(0)
        { string s=NULL;
          void *p=(void*)&s; 
          BP_get_simple(BP_input,p);
          if(NULL==s) BP_fail();
          printf("got string = %s\n",s);
          BP_result=BP_atom(s);
        }
        else { 
          double d;
          void *p=(void*)&d; 
          BP_get_simple(BP_input,p);
          printf("got double = %f\n",d);
          BP_make_float(BP_result,d+1);
        }        
      } 
      break;
      
      /*
        getts a list of known length, applies
        to the a function of type int f(void **) 
        of simple args and returns an int
        a list like [hello,3.14,2003] will
        need argc=0 and argv should have the address
        of variable of APPROPRIATE types. Conversion
        of data is automatic but we trust YOU to provide
        the right containers in C !!!
      */
      case 17: {
        string s;
        double d;
        int i,n;
        int argc=3;
        void *argv[3] ={(void*)&s,(void*)&d,(void*)&i};
        if(NULL==BP_get_list(BP_input,argc,argv)) BP_fail();
        printf("got from prolog list: %s, %f, %d\n",s,d,i);
        BP_result=BP_integer(i);
      } 
      break;
      
     /* EDIT AND ADD YOUR CODE HERE....*/

    default:
     return LOCAL_ERR(X(1),"call to unknown user_defined C function");
  }
  return H; 
}

\end{verbatim}
}

\subsection{Calling Prolog from C}

Normally this is done after a first call from Prolog to C
(this gives a chance to the prolog system to get initialized).
The most flexible technique is based on multiple engines. Take a look
at {\tt case 11} and {\tt case 12} in the previous section.

The lower-level interface, (which follows) is still usable
but less recommended.

{\small
\begin{verbatim}
/* this can be used to call Prolog from C : see example if0 */

term bp_prolog_call(goal,regs,H,P,A,wam)
  register term goal,regs,H,*A;
  register instr P;
  register stack wam;
{
  PREP_CALL(goal);
  return bp(regs,H,P,A,wam);
}

/* simple example of prolog call */
term if0(regs,H,P,A,wam)
  register term regs,H,*A;
  register instr P;
  register stack wam;
{ term bp();
  cell goal=regs[1];

  /* in this example the input GOAL is in regs[1] */
  /* of course you can also build it directly in C */
  /* unless you want specific action on failure,
     use BP_prolog_call(goal) here */

  H=bp_prolog_call(goal,regs,H,P,A,wam); 
  if(H) 
     fprintf(stderr,"success: returning from New WAM\n");
  else 
     fprintf(stderr,"fail: returning from New WAM\n");

  /* do not forget this !!! */
  return H; /* return NULL to signal failure */
}
\end{verbatim}
}

BinProlog's {\tt main()} should be the starting point of your program
to be able to initialize all data areas. To call back from C you
can follow the example if0. A sustained BinProlog-C dialog
can be set up by using the 2 techniques described previously.

An example of using the C-interface (composed of files c.h c.c c.pl)
can be found in directory {\tt c\_inter}.
 If you whish you can create
a standalone C-executable by using BinProlog's compilation to C
(see directories {\tt pl2c}, {\tt dynpl2c}) you can
simply type `make' in directory  {\tt c\_inter}.

\section{A 3-tier BinProlog/C/Java interface through JNI}


Simple BinProlog Java interface for Windows, Solaris and Linux.

\subsection{Overview}

The BinProlog runtime emultor combined with the C-ified
compiler are packaged into a dynamic library (jbp.dll or libjbp.so).
A stub jBinPro.c based on BinProlog's C interface
implements a call\_bp\_main C function which is
declared as a native Java method in file JavaLog.java. 

A makefile (you might whish to edit for path information)
takes care of the whole process.

The scripts jbp.bat (Windows) and jbp (Solaris)
can be used to run the resulting application.
You might have to edit them for path information.

More work, BinProlog Professional source license and possibly the sources
of the JDK kit from Sun are needed to make this into an applet runnable
from Netscape. Note also that currently native methods in Java are of
limited use due to security concerns and that such an applet might be
restricted to your INTRANET.

I would suggest using BinProlog's Linda-based Jinni extension
for unrestricted INTERNET applications or a combination of Java
and BinProlog components.

Still, for a minimal overhead interaction, in case your interface
needs to glue together C/C++ Prolog and Java components, the effort
might be worth it. On platforms where BinProlog does not support
multi-threading, using Jinni's (a simplified Java based Prolog dialect)
might be particularly appealing.

\subsection{The Prolog Component}

This example test bidirectional BinProlog/Java connection going through
BinProlog's C interface. The Java application is Jinni, which 
also provides Prolog functionality. These can be seen also as an interface
between a Prolog-in-Java and a Prolog-in-C.

\begin{verbatim}
main:-
  write('BinProlog called from Java'),nl,
  % WILL CALL Jinni services as follows:
  % ask_jinni(Pattern,Goal,Answer) Answer=the(Pattern') | no
  write('calling back Java using new_bultin, with:'),nl,
  In='listing(append)',
  write(In),nl,
  new_builtin(1,In,Out),
  write('back to Prolog again'),
  write(Out),nl,
  !.
main:-
  write('something failing in main.pro'),nl.
\end{verbatim}

\subsection{The Java Component}

This is the master entry point of the application as BinProlog 
(written in C) is seen as a Java native method (using JNI, JDK 1.1 convetions).
When started with {\tt java jbp} its {\tt main} method will load the
native method obtained by linking jbp.c with BinProlog object files.
Then it calls BinProlog's main entry C function with main.pro as parameter.
It also provides a Java method {\tt fromC} to be called back
(this is supported by JNI 1.1) later.

\begin{verbatim}
class jbp {
  public native String call_bp_main(String jGoal);
  static {
    System.loadLibrary("jbp");
  }
  
  public  String fromC(String k) { // constructor
                  
         System.out.println("back from java:"+k);
              
         return "CallFromCtoJava, "+k;
  }

  public static void main(String args[]) {
    jbp a_jbp=new jbp();
    String Answer=a_jbp.call_bp_main("$main.pro"); 
    System.out.println("ANSWER:"+Answer);
  }
}
\end{verbatim}

\subsection{The C component}

The C component is a based simplified version of {\tt new\_builtin} from 
directory {\tt c\_inter}, showing calls from BinProlog to C and back.

To make some sense out of this, the reader has to start by understanding
the JNI from the documentation 
at www.javasoft.com and paly first with our C-interface described
in the previous section.

{\small \begin{verbatim}
#include <jni.h>
#include "jbp.h"
#include <stdio.h>

/* code BinProlog will call before after its own initalization process */

int init_c() {
  printf("initialising user's C code\n");
  return 1;
}

/* MAIN FILE: change this if you want to call BinProlog as a DLL */
/* or some other form of dynamically linked library              */

/***** BinProlog's embedding as a Java native procedure      *****/

/*
 * Class:     jbp
 * Method:    call_bp_main
 * Signature: (Ljava/lang/String;)V
 */

static JNIEnv *jenv=NULL;
static jobject jobj=NULL;

JNIEXPORT jstring JNICALL 
Java_jbp_call_1bp_1main (JNIEnv *env, jobject obj, jstring jGoal) {
	int argc=2; char *argv[]={"bp","$main.pro",NULL};
	extern int bp_main(int argc, char **argv);
  argv[1]=(*env)->GetStringUTFChars(env, jGoal, 0);
  jenv=env;
  jobj=obj;
	(void) bp_main(argc,argv);
  return jGoal;
}

/****************************************************************
          STUB IN CASE THE C_INTERFACE IS NOT USED

(Real) files using new C-interface (c.pl c.h c.c have been moved to 
directory ../c_inter
*****************************************************************/

#include "global.h"

extern struct specsyms g;

string c_interface="with Java and C Interface";

static string c_errmess="c.c: prototype Java interface\n";

term new_builtin(register term H, register term regs, register term *A,
                 register instr P, register stack wam)
{
  jstring jback,jto;   const char* cback;  const char* csend="FromCtoJ";
  jclass cls = (*jenv)->GetObjectClass(jenv, jobj);
  jmethodID mid = (*jenv)->GetMethodID(jenv, cls,"fromC", 
	  "(Ljava/lang/String;)Ljava/lang/String;"
  );
  
  fprintf(STD_err,c_errmess);

  
  if (mid == 0) return NULL;
  jto=(*jenv)->NewStringUTF(jenv, csend);
  jback=(*jenv)->CallObjectMethod(jenv, jobj, mid,jto);
  cback=(*jenv)->GetStringUTFChars(jenv, jback, 0);
  printf("BACK!%s\n",cback);   
  
  /*return NULL;  always fails */ 
  regs[1]=regs[4];
  *H=regs[2];
  regs[3]=(cell)H++;
  return H;  
}

term if0(register term regs, register term H, register instr P, 
         register term *A, register stack wam)
{ /* unused builtin */
  fprintf(STD_err,c_errmess);
  return NULL;
}
\end{verbatim}}

\section{BinProlog's Tcl/Tk interface (written in cooperation with Bart Demoen, KU. Leuven, back in 1993 or 1994)}

This interface is based on and abstract scheme for the embedding of two languages with
strong meta-programming, parsing and structure manipulation
capabilities.  We provide a call-back mechanism requiring each language
to {\em compile} its service requests to the other's syntax for processing
through the other's {\em meta-programming}
capabilities. {\em State mirroring} ensures
that only differential information needs to be transmitted.  We
describe a practical implementation of these principles and our
experience with the {\bf BinProlog Tcl/Tk} interface. 
Compilation by each side, that targets
the other's syntax and dynamic object manipulation capabilities,
achieves in {\em less than two hundred} 
lines of code, a fully functional integration
of the two systems. The environment
is portable to any Prolog system to which it
adds powerful graphic and transparent distributed
network programming facilities.
Our approach suggest that logic programming language-processors
can be made to fit easily in a multi-paradigm programming environment
with little programming effort by using their
strong {\em language translation} and {\em meta-programming}
capabilities.

\subsection{Introduction}

Integration of various programming paradigms is a necessity
in modern programming environments. 
Monolithic language implementations tend to be replaced more and more
by a combination of specialised {\em tools}.
In particular, generic graphic processors can be connected
to various languages which do not need 
their own graphic primitives anymore.

However, when there's a large
semantic distance between the languages and they feature
radically different computational models, a low-level
interface is a painful and not always rewarding programming
task. It happens very often that when the interface is finished
one of the sides of the interface just becomes obsolete
because of
rapid evolution. This is especially true with various
windowing systems which are often machine/operating system dependent.
 
The aim of this paper is to propose an unusually quick
interfacing technique which uses meta-programming capabilities
on the two sides. The basic idea is to link the two languages
through a standard client-server interface and have each of
them drive the other as an interactive agent, by generating
the appropriate code on the fly.
By mirroring the state of the objects, only state
changes have to be transmitted from one side to the other,
so that the granularity of the interaction will not become
a bottleneck. Moreover, if one side has the capacity to
react to events, this property will be inherited with
minimal programming effort by the other side.

This paper is motivated by our work on finding a simple and
portable way to add a visual programming environment to
Prolog systems.

The popular Tcl/Tk visual language by John Ousterhout \cite{Oust:TclTk}
is basically a composite made of
of a shell-like interpreted programming
language (Tcl) and a high-level Motif-style graphic
package (Tk). Tcl is an untyped string-only language with
strong meta-programming facilities (i.e. an {\em eval} primitive
and dynamic procedure creation).

After describing a general language-processor interaction
model we will report
how it has been applied to an interface between BinProlog \cite{Tarau93:GULP},
and Tcl/Tk and how this interface has been ported to
another Prolog system (Prolog by BIM) with minimal
programming effort.

\subsection{An abstract multi-language communication model}

Let {\tt L} and {\tt R} be two languages, $t_{L,R}$ a translation function
(compilation) from L to R, $t_{R,L}$ the reverse translation from
{\tt R} to {\tt L}.
Let $e_L$ and respectively
$e_R$ be the {\em eval} operations of {\tt L} and {\tt R}.
Let $r_L$, $w_L$, $r_R$ and $w_R$ denote the read and write operations
of languages L and R.

An $L_R$-processor is a process with capabilities to
\begin{enumerate}
\item execute programs written in L
\item implement a translation function from L to R.
\end{enumerate}

We say that two language processors $L_R$ and $R_L$ are {\em connected}
if the following conditions are verified:
\begin{enumerate}
\item each {\em translate+write} operation on one side triggers exactly one
{\em read+eval} operation on the other side,
\item one or more\footnote{The ability to trigger more than one {\em translate+write}
is needed for instance when more than one task has to be
initiated or more than on component of the state of one side
has to be updated in a transaction.
} {\em translate+write} operations
can be triggered by each {\em eval} operation.
\end{enumerate}

{\flushleft This}
can be formalized more precisely in terms of temporal modal
operators. 

Our {\em write} and {\em read}
operations should not be confused with
ordinary input-output primitives. More precisely they are abstracted
from the subset of input-output operations which can be meaningfully
translated and are therefore suitable
as input for the other side's eval functions.
Thus they can be seen as carrying messages from an $L_R$
processor to an $R_L$ processor.

We say that an $L_R$ and an $R_L$ processor
are {\em fairly interacting} if
\begin{enumerate}
\item each write operation on one side will be 
eventually read on the other side
\item
the behaviour of
each side is observable on the other side in terms of a sequence of successive
read and write operations
\end{enumerate}

If for two $L_R$ and $R_L$ language processors are {\em connected}
and their read and write operations are queued, then
they are {\em fairly interacting}.

We will break this fully symmetric behaviour by defining
a {\em slave} language-processor as one that passively waits
for write operations from the other side and a {\em master}
language-processor which reacts to a write operation
by a corresponding read operation with
its result passed to its {\em eval} operation.
We suppose also that the user interacts only with the master.
As only the slave waits blocked on read operations, the
master is free to be `multi-threaded' or event-driven.

\subsection{The BinProlog to Tcl/Tk interface}

As an instance of this general scheme,
we have chosen to connect BinProlog as a `slave process'
to a Tcl/Tk {\em wish} shell enhanced
with reactive capabilities
given by the {\em addinput} facility~\footnote{
A modification of the Tcl/Tk event loop (
freely available from
ftp.neosoft.com)
which enables it to react to the presence of new input
by triggering the execution of a user specified procedure
when input becomes available.}. A specialized BinProlog
toplevel is launched from the Tcl/Tk shell as a
background Unix process connected through
a bidirectional pipe.
The BinProlog process is able to generate and react to Tcl/Tk events.

The interface uses tcl7.3 with tk3.6 combined with addinput-3.6a.  It
consists of 91 lines
of Prolog and 101 lines of (commented) Tcl code.  Thanks to the
strong meta-programming capabilities of both languages the interface
boldly {\em compiles} 
messages from one side
to evaluable representations on the
other side and calls the appropriate
{\em eval} operation.
The interface is portable to other
Prolog systems and requires no C-programming. 
The installation of a BinProlog slave on the 
`event channel' handled by {\tt addinput} is done as follows:

{\small \begin{verbatim}
proc start_prolog {} {
#   Opening a bidirectional pipe to BinProlog
#   under control of the addinput facility
    set f [open "|bp -q5 server.bp" r+]
    addinput $f "bpInOut %% %E %F"
}
\end{verbatim}}

{\flushleft Output} from the BinProlog slave
{\tt bp server.bp} activates a Tcl/Tk
procedure which executes on a {\em new} Tk-interpreter without
disturbing normal interaction with the {\em wish} shell.

Basically, performance is not affected by the parsing
as this is kept minimal. Moreover in systems with a parser
written in C
(as Prolog by BIM or the Koen De Bosschere's 
ISO-parser based version of BinProlog)
parsing is in itself fast enough not to dominate
the interaction cost.
Techniques as {\em state mirroring} which will be discussed in the next
section are also used to minimize the communication overhead.

High-level primitives on the Tcl/Tk side ensure that
output to Prolog are valid Prolog terms.
On the Prolog side a special toplevel loop reads the
Tcl/Tk message, and applies its eval operation
(call/1) to it. To avoid spurious interaction due
to syntax errors or unexpected messages on each side
only lines having a reserved header (i.e. 
\verb~call_prolog~
and respectively \verb~call_tcl~)
are evaluated. However, to give the look and feel of interacting
with a Prolog system, messages from Prolog which are not
of the form \verb~call_tcl {...}~ will be printed out
as such. This fits well with Tcl which is string oriented
but would not fit well with a term-oriented language like
Prolog without parser modifications.

\subsection{State mirroring}

State mirroring is a simple technique that consists of
duplication by a given language processor of (relevant)
state information of an interactively connected other
language processor.

In the case of Prolog, we have chosen to avoid the
representation of states by infinite forward loops
for two reasons:

\begin{enumerate}
\item unexpected failure would make the system unreliable
and hard to debug
\item unexpected delays induced by garbage collection would make
a visual interface unpleasant to use
\end{enumerate}

Fortunately, BinProlog's blackboard turned out to be the perfect match
to mirror the state
of various Tcl/Tk objects. A library emulating BinProlog's blackboard
operations in terms of generic Prolog 
dynamic database operations
has been used to achieve
the same effect in the Prolog by BIM port of the interface.
 
\subsection{Programming with the embedded environment}

Using the BinProlog-Tcl/Tk interface is very simple.
Let \verb~<P>~ be the name of the composite program.
Suppose the user creates 2 files \verb~<P>.tcl~ and 
\verb~<P>.pl~. The composite `program' can be executed
as follows:

{\small \begin{verbatim}
    $ wish
    % source <P>.tcl
\end{verbatim}}

{\flushleft \verb~<P>.tcl~ is expected} to have at its end something like:

{\small \begin{verbatim}
   start_prolog
   p {compile(<P>)}
\end{verbatim}}

{\flushleft The} user will get the usual Tcl command prompt to interact with
either Tcl or Prolog. A command entered at the Tcl prompt
normally `goes to' Tcl except for the following: 

{\small \begin{verbatim}
    start_prolog      -connects to a new Prolog process
    halt_prolog       -terminates the Prolog process:   

    p {<PrologTerm>}  -sends a goal to Prolog for quiet evaluation

    q {<PrologTerm>}  -sends a query to Prolog and gets back answers
\end{verbatim}}

{\flushleft After} being activated by one of the previous (p or q) Tcl commands
a Prolog goal may `call back' to continue the dialog with Tcl/Tk using
the following Tcl command on the Prolog side:

{\small \begin{verbatim}
   call_tcl/1    -sends to Tcl a command to be immediately
                  evaluated in `background', while the Tcl shell
                  prompts and waits for Tcl or Prolog commands.
\end{verbatim}}

{\flushleft Output} with {\tt puts} (Tcl) or {\tt write/1}
(Prolog) goes as usual to the {\tt stdout} of the Tcl shell.

\subsection{A visual N-queens program}


This program illustrates two simple-minded but useful features
of the interface:

\begin{enumerate}

\item  Prolog lists are sent to Tcl as Tcl-lists
\item  Prolog simply pumps solutions one by one using their most natural
   representation through \verb~call_tcl(display_queens(L))~.

\item synchronization is done as follows:
\begin{itemize}
\item the user provides mouse events
generating write operations 
\item the prolog process waits for input on a read operation
until asked for another answer
\end{itemize}
\end{enumerate}

Equivalently, the user
can also control the Prolog program from Tcl's command line
by typing in "p more" or "p done" messages.

The code on the Prolog side consists of a
classic N-queens program, the generic
prolog interface program server.pl
and the clause:

{\small \begin{verbatim}
qs(N):-
    queens(N,L),
    call_tcl(display_queens(L)),   % sends a display request
    write('type <p done> when finished <p more> otherwise'),nl,
    tcl_in(call_prolog(done)),!.       % waits for `done' or `more'
\end{verbatim}}

{\flushleft The} code on the Tcl/Tk side
consists of the generic server.tcl interface program,
routines to set up the visual display and procedures like
\verb~show_queen~, \verb~hide_queen~ which are called by the
main Tcl/Tk-side routine:

{\small \begin{verbatim}
proc display_queens {qs} {
  global w count
  incr count
  hide_queens
  set l 0
  foreach q $qs {
    incr l
    show_queen $w [expr $l -1] [expr $q -1]    
  }  
}
\end{verbatim}}

{\flushleft This} procedure simply counts the answers and updates the display
when a new answer comes from Prolog, triggered by the presence
of new input on the pipe.

% \subsection{A visual heap-measurement tool}

\subsection{Performance evaluation}

Tcl/Tk is itself a glue language which links together
large blocks of C-code.
We have written a naive-reverse program
and measured that it executes in Tcl between
20-50 times slower than in BinProlog.
This definitely suggests that our technique, using
relatively expensive parsing and compilation operations
will not become itself an interaction bottleneck and
therefore due to the much
higher flexibility and almost instant
portability between various Prolog systems
this approach looks superior to a highly
Prolog-specific C-level integration.

As BinProlog is a much faster symbolic processor than Tcl,
we have
decided to do most of the translation
operations on the Prolog side.
We have measured about 1500 BinProlog-Tcl/Tk exchanges per second
on a Sparcstation 2 and 700-800 exchanges on a Sparcstation 1,
both serving the same Tektronix X-terminal through a network.

This confirms that the interface is fast enough for complex
real time visualisation tasks and pleasant to use in a
average Unix environment.


\subsection{Porting the interface to another Prolog}

It took one hour to make the interface work with ProLog by BIM.
The main change was to modify \verb~start_prolog~ on the
Tcl side to start a ProLog by BIM process with:

{\small \begin{verbatim}
   set f [open "|bim server.pro" r+]
\end{verbatim}}

{\flushleft  The} ProLog by BIM process is initialized from
a small new toplevel \verb~server.pro~. The one hour work
had more to deal with the differences between the two 
Prolog systems,
than the interface itself. Moreover, code written for
one Prolog system (like the {\tt N-queens} program) 
runs without changes on the other, 
as far as it is written in portable Prolog.
We think this compares favorably with
a dedicated piecewise interface written
in C for each Prolog graphic builtin
in terms of programming effort,
portability and learning curve for the user,
and, as our performance evaluation has shown,
with reasonable real-time interactivity.

\subsection{Related work}

Another Prolog with Tcl/Tk interface we know
of is that developed by
Micha Meier at ECRC as the standard
Eclipse GUI. It has been ported to SICStus Prolog as well.
Micha Meier's interface is tightly integrated (at C-level)
and Eclipse specific (although a Sicstus port is also provided).
The tighter integration is better (in principle)
in the case
of a very low granularity communication, although
the programming effort to implement it is
more considerable than ours.
Our performance analysis
shows that global costs are dominated by process switching,
graphic manipulations and brute force processing in Prolog.
Moreover, our interface, does
not rely on a specific Tcl/Tk implementation and has
no modification to the C-code neither on the Prolog emulator
neither on the Tcl/Tk side.
Because of this reasons, we believe that
the simplicity and portability
of our design is a clear advantage,
especially in the context of rapidly evolving
backward incompatible Tcl/Tk releases.

\section{Projected future work on the interface}
We plan to make the interface fully Prolog-independent
and to write Prolog libraries to support various
existing Tcl/Tk extensions and tools for
distributed Tcl/Tk programming,
automatic interface generators etc.
A Tcl/Tk based generic Prolog toplevel would also
be very helpful to give the full illusion of
{\em being in Prolog} while benefiting of
visual goodies of a Tcl/Tk shell.

\subsection{Conclusion on the interface}

We have outlined a generic programming framework for a seamless
integration of two programming languages which have `eval' capability
and implemented an instance of the framework as a Prolog to TCL/Tk
interface.

Our experiments show that
support for metaprogramming tools
in modern programming environment
is important not only for the expressiveness of the
language itself but also multi-paradigm language
integration.

We have shown that with a reduced programming effort
a powerful interface can be built between a Prolog system and
a state-of-the art visual environment. 

This suggests that embedding of a logic engine in a
generic multi-language
environment is competitive in functionality and performance
with the traditional approach which advocates to hardwire
in the Prolog system a set of language-specific extensions.

The code for the interface can be obtained by ftp from
{\bf clement.info.umoncton.ca} and has been integrated in the
BinProlog starting with the 2.20 distribution.

\section{Interfacing with Internet Languages}

%\subsection{CGIs}

%\subsection{Generating VRML}

%\subsection{BinProlog/C/Java 3-tier socket based communication}

Our approach to multi-language/multi-machine communication 
through the net has tried to keep the same touch of simplicity
as the previously described Tcl/Tk interface \cite{TD94:WE}.
Instead of exposing low-level C operations, we have chosen
to build more convenient abstractions.

Related BinProlog documentation is available at: 
\cite{bp7user,bp7advanced,bp7interface,bp7crossref}.

Our recent Jinni and BinProlog related papers are available
from:

\begin{verbatim}
  http://www.cs.unt.edu/~tarau/
\end{verbatim}

Related publications: \cite{T98:Jinni,ilps97tut,tdb95rev,TD94:WE}
 
\end{document}

